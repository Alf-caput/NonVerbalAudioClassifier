\section{Estado de la cuestión}
\subsection{Investigaciones existentes}
\subsubsection*{Artículo 1}
\begin{flushleft}
    \textbf{Título:}
    “Desarrollo y análisis de clasificadores de señales de audio”
    \\
    \textbf{ID: \cite{aguirre2017desarrollo}}
    \\
    \textbf{URL:}
    \url{https://riunet.upv.es/handle/10251/90005}
    \\
    \textbf{Resumen:\\}
    En este artículo podemos encontrar un análisis sobre la clasificación y reconocimiento de señales de audio, donde podemos destacar su relevancia en aplicaciones modernas que combinan procesamiento de audio y aprendizaje automático. En este artículo se mencionan temas bien establecidos como el reconocimiento de voz y la música, así como un campo emergente, la clasificación automática de ruidos ambientales. Para el aprendizaje automático emplea un modelo supervisado, donde los datos de audio están etiquetados en clases. Se describe un proceso de clasificación, que abarca desde la extracción de características del audio hasta la aplicación de algoritmos de aprendizaje automático para esta tarea. También, se proporciona un sistema de clasificación y etiquetado de audio en tiempo real como ejemplo concreto de aplicación práctica, incluyendo una estructura de clasificación jerárquica.
    \\
    \textbf{Por qué es útil:\\}
    Esta información puede ser muy útil para orientar nuestro proyecto en el desarrollo de un modelo que diferencie audios de tos, estornudos y risas. También es importante para nuestro proyecto la extracción de características del audio y el uso de algoritmos de aprendizaje automático para la clasificación. Además, se menciona un enfoque supervisado, que es importante cuando tienes datos etiquetados en clases. Por último, el artículo menciona el desarrollo de un software para la clasificación de audio en tiempo real, lo cual puede ser útil para implementar el modelo en un entorno práctico.
\end{flushleft}
\subsubsection*{Artículo 2}
\begin{flushleft}
    \textbf{Título:}
    “Environmental sound recognition: a survey”
    \\
    \textbf{ID: \cite{chachada2014environmental}}
    \\
    \textbf{URL:}
    \url{https://www.cambridge.org/core/journals/apsipa-transactions-on-signal-and-information-processing/article/environmental-sound-recognition-a-survey/96211C365DC7B250CEEFFB15026A5CDF}
    \\
    \textbf{Resumen:\\}
    El artículo ofrece una revisión detallada y comprensible de los avances más recientes en el campo del reconocimiento de sonidos ambientales, organizada en cuatro secciones principales. Estas incluyen una explicación de los enfoques básicos de procesamiento de sonidos ambientales, técnicas para manejar sonidos ambientales estacionarios y no estacionarios, así como una comparación del rendimiento entre diferentes métodos. Por último, el artículo discute las conclusiones obtenidas y señala las tendencias futuras en la investigación y desarrollo de este campo específico.
    \\
    \textbf{Por qué es útil:\\}
    Este artículo es útil porque proporciona revisión detallada de los avances de campo del reconocimiento de sonidos ambientales. Al revisar las técnicas presentadas en el artículo, podrás identificar técnicas relevantes que podrían ser aplicables a tu proyecto de diferenciar entre audios de tos y risa. Por ejemplo, podrías encontrar métodos para manejar aspectos no estacionarios de los sonidos, lo cual puede ser relevante para identificar características distintivas de la tos y la risa en los audios.
\end{flushleft}
\subsubsection*{Artículo 3}
\begin{flushleft}
    \textbf{Título:}
    “Clasificación automática de sonidos utilizando lenguaje máquina”
    \\
    \textbf{ID: \cite{rodriguez2020clasificacion}}
    \\
    \textbf{URL:}
    \url{https://idus.us.es/handle/11441/101412}
    \\
    \textbf{Resumen:\\}
    Este artículo realiza la implementación de dos modelos para clasificar sonidos ambientales mediante aprendizaje automático. Ambos resaltan la importancia del procesamiento del espectrograma de audio y el preprocesamiento de datos. La finalidad de este artículo es crear un sistema para clasificar sonidos ambientales con precisión aceptable, explorando métodos de extracción de características y utilizando inteligencia artificial.
    \\
    \textbf{Por qué es útil:\\}
    Este trabajo es útil ya que aborda un tema muy conectado al nuestro, nuestro objetivo es la clasificación de sonidos producidos por las personas. Aunque este artículo clasifique sonidos ambientales nos puede ayudar a relacionar el enfoque que podemos darle a nuestro trabajo.Además utiliza la transformada de fourier, luego también es útil para empezar a conocerla para poder trabajar con ella.
\end{flushleft}
\subsubsection*{Artículo 4}
\begin{flushleft}
    \textbf{Título:}
    “Reconocimiento automático de instrumentos mediante aprendizaje máquina”
    \\
    \textbf{ID: \cite{salgado2019reconocimiento}}
    \\
    \textbf{URL:}
    \url{https://idus.us.es/handle/11441/91372}
    \\
    \textbf{Resumen:\\}
    Este proyecto aborda el reconocimiento de instrumentos musicales mediante múltiples algoritmos de aprendizaje automático, incluyendo KNN, redes neuronales, PCA, LDA y Random Forest. Se evalúan distintos enfoques y se concluye que, en algunos casos, los algoritmos más simples, como KNN, ofrecen resultados óptimos. Se destaca la importancia de la reducción de dimensiones y se exploran las dificultades asociadas con instrumentos específicos.
    \\
    \textbf{Por qué es útil:\\}
    La utilidad de la información del estudio de instrumentos se refleja en la experiencia adquirida en el procesamiento de datos de audio y en la selección de algoritmos eficaces para la clasificación, realizando un contraste de estos para elegir el mejor candidato. Además, conceptos como la reducción de dimensiones y la adaptabilidad a contextos cambiantes pueden aplicarse al reconocimiento de sonidos en nuestro proyecto.
\end{flushleft}
\subsubsection*{Artículo 5}
% \begin{flushleft}
%     \textbf{Título:}
%     \\
%     \textbf{ID: \cite{}}
%     \\
%     \textbf{URL:}
%     \url{}
%     \\
%     \textbf{Resumen:\\}
%     \\
%     \textbf{Por qué es útil:\\}
% \end{flushleft}
\subsubsection*{Artículo 6}
% \begin{flushleft}
%     \textbf{Título:}
%     \\
%     \textbf{ID: \cite{}}
%     \\
%     \textbf{URL:}
%     \url{}
%     \\
%     \textbf{Resumen:\\}
%     \\
%     \textbf{Por qué es útil:\\}
% \end{flushleft}
\subsubsection*{Artículo 7}
% \begin{flushleft}
%     \textbf{Título:}
%     \\
%     \textbf{ID: \cite{}}
%     \\
%     \textbf{URL:}
%     \url{}
%     \\
%     \textbf{Resumen:\\}
%     \\
%     \textbf{Por qué es útil:\\}
% \end{flushleft}
\subsubsection*{Artículo 8}
% \begin{flushleft}
%     \textbf{Título:}
%     \\
%     \textbf{ID: \cite{}}
%     \\
%     \textbf{URL:}
%     \url{}
%     \\
%     \textbf{Resumen:\\}
%     \\
%     \textbf{Por qué es útil:\\}
% \end{flushleft}
\subsection{Tabla resumen}
