
\section{Resolución}
\subsection{Organización}
Organización temporal y de trabajo

\subsection{Desarrollo}
Desarrollo y resolucion completa
\subsection{Implementación}
Para la implementación del proyecto, se ha desarrollado una aplicación que ofrece una solución práctica para la detección en tiempo real de varios tipos de sonidos, como tos, aclaración de garganta, risas, suspiros, esnifar y estornudos.

La aplicación se inicia importando los módulos necesarios y configurando la detección de los tipos de sonidos, además de cargar un modelo previamente entrenado con una efectividad del 85\%. Luego, utiliza Matplotlib para mostrar las formas de onda de audio en tiempo real y capturar las muestras entrantes para su procesamiento. Mediante una función definida, la aplicación detecta los diferentes tipos de sonidos utilizando el modelo cargado y los presenta en un gráfico para una fácil comprensión.

El flujo de audio se inicia mediante SoundDevice y se crea una animación continua para visualizar de manera fluida las muestras de audio y sus clasificaciones. Esta implementación puede ser muy útil en diversos ámbitos profesionales, como en las llamadas telefónicas (para evaluar el nivel de satisfacción del cliente durante la llamada) y en servicios sanitarios (para supervisar el bienestar del paciente).