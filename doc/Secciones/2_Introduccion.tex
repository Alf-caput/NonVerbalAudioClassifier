
\section{Introducción}
En la era digital actual, la capacidad de las máquinas para interpretar y responder a nuestro entorno ha transformado diversas áreas de investigación y desarrollo tecnológico. Aquellas campos de investigación que aprovechan las posibilidades que brindan las nuevas tecnologías de inteligencia artificial y aprendizaje automático, están experimentando un crecimiento exponencial.

 La clasificación y reconocimiento de sonidos, especialmente en contextos sociales y de salud, está siendo un campo de gran interés debido a sus amplias aplicaciones prácticas. Desde la monitorización de condiciones en pacientes hasta la interacción mejorada entre humanos y dispositivos inteligentes. En la actualidad, casi la totalidad de dispositivos tecnológicos que salen al mercado incorporan asistentes de voz inteligentes capaces de entender el habla humana. Tomando en cuenta estos hechos, se puede concluir que  el análisis automático de sonidos ofrece prometedoras avenidas para mejorar la calidad de vida y la eficiencia operativa en múltiples dominios.

Dentro de este área de interés, los sonidos humanos que clasificamos como “no-verbales” tales como la tos, las risas, los llantos y los suspiros son indicadores vitales de diversas condiciones; tanto anímicas como físicas y de salud . En entornos como hospitales y residencias de ancianos, la identificación precisa de estos sonidos puede ser esencial para la provisión de atención y bienestar que asegure unos cuidados de calidad a grupos de población más vulnerables. Sin embargo, la detección y clasificación eficaz de dichos sonidos en entornos ruidosos y no estructurados sigue presentando un desafío técnico significativo debido a diversos factores como las condiciones ambientales desfavorables y las limitaciones de software actuales.

Con lo cual, este proyecto propone desarrollar y utilizar algoritmos avanzados de aprendizaje automático para la clasificación precisa de sonidos específicos en diferentes entornos sociales para su posterior implementación. A través del uso de técnicas como el procesamiento de señales y redes neuronales profundas, se busca crear un sistema capaz de identificar de manera fiable sonidos humanos , haciendo una diferenciación clara de otros ruidos de fondo y asegurando una respuesta contextual apropiada.

Tras muchas consideraciones , se ha determinado que este proyecto podría ser de gran  importancia, pues tiene un amplio potencial para impactar positivamente en la sociedad; especialmente en el ámbito de la salud. Por ejemplo, en hospitales, la detección automática de sonidos como la tos o el llanto puede ayudar al personal médico a monitorizar más efectivamente el estado de los pacientes, permitiendo intervenciones más rápidas y adecuadas. 
Asimismo, en otros ámbitos como el del hogar inteligente, este sistema puede mejorar la interacción entre los usuarios y sus entornos; facilitando un ambiente más reactivo y sensible a sus necesidades.Un ejemplo de ello sería el ajuste automático de la temperatura y humedad del hogar al detectar un elevado nivel de toses o estornudos.
En conclusión, la identificación y clasificación de sonidos no verbales no solo es un desafío técnico, sino que también surge como una necesidad emergente que está trayendo avances significativos para la comodidad y el bienestar de la población.
\subsection{Objetivos}
\subsubsection{Objetivo general}
Identificar y extraer diferentes tipos de sonidos a partir de fragmentos de audio para su posterior análisis e implementación en diferentes modelos sociales. Para ello aplicaremos la Transformada de Fourier para limpiar el sonido de nuestros fragmentos de audio. Posteriormente implementaremos un modelo de clasificación, el cual entrenaremos y haremos predicciones con nuevos datos.
\subsubsection{Objetivos específicos}
\begin{enumerate}
    \item \textbf{Explorar y adquirir datos de audio:} Recopilar una amplia variedad de muestras de audio que representen diferentes tipos de sonidos relevantes para el análisis y modelado social.
    \item \textbf{Preprocesamiento de datos de audio:} Utilizar herramientas de procesamiento de señales para preprocesar los datos de audio, realizando la normalización del volumen y la eliminación de ruido de fondo no deseados. Este último lo realizaremos mediante la aplicación de la transformada de fourier para representar la señal en el dominio de la frecuencia, analizar este espectro de frecuencia para identificar las componentes de sonido deseadas y el ruido no deseado y por último realizar una eliminación selectiva de frecuencias no relevantes para mejorar la calidad de la señal y reducir el impacto del ruido.
    \item \textbf{Implementar la Transformada de Fourier:} Utilizaremos la biblioteca \textit{numpy} de Python, que permitirá transformar los datos de audio del dominio temporal al dominio de la frecuencia.
    \item \textbf{Análisis de frecuencia de sonido:} Analizar los espectros de frecuencia resultantes para identificar las características espectrales distintivas de diferentes tipos de sonidos, como tonos, frecuencias dominantes y distribuciones de energía en diferentes bandas de frecuencia.
    \item \textbf{Extracción de características relevantes:}Identificar y extraer características relevantes de los espectros de frecuencia que puedan ser útiles para la clasificación y análisis de diferentes tipos de sonidos, como la amplitud máxima, la frecuencia dominante o  la dispersión espectral.
    \item \textbf{Desarrollo de modelos de clasificación:}Utilizaremos técnicas de aprendizaje automático, como redes neuronales, para desarrollar un modelo que pueda clasificar automáticamente los diferentes tipos de sonidos basados en las características extraídas.
    \item \textbf{Validación y evaluación del modelo:}Evaluaremos el rendimiento de los modelos de clasificación con nuevos datos para evaluar el rendimiento de nuestro modelo.  Utilizando métricas como la precisión.
    \item \textbf{Implementación y despliegue del modelo:}Integraremos el modelo de clasificación final en una aplicación para su implementación práctica (de manera únicamente teórica), lo que llegará a poder realizar una detección y clasificación a tiempo real en entornos sociales específicos.
\end{enumerate}
\subsection{Alcance e impacto}
Nuestro equipo compuesto por cuatro ingenieros matemáticos se enfrenta al objetivo de desarrollar un modelo de aprendizaje automático, para clasificar audios con diferentes sonidos como tos, risa, estornudos… etc; dado que no disponemos de un presupuesto asignado, utilizaremos recursos gratuitos y códigos abiertos para llevar a cabo el proyecto.

Los objetivos propuestos por el equipo del proyecto tratan del desarrollo de un modelo de aprendizaje automático, esto conllevará  la recopilación de datos, la selección de características relevantes de los audios, el entrenamiento del modelo y la comprobación de su rendimiento, por último se hará un análisis de los resultados obtenidos para sacar conclusiones sobre las aplicaciones del modelo que puedan llegar a generar un impacto significativo, además identificaremos posibles mejoras para futuras iteraciones del proyecto.

Las fechas de los objetivos estarán bien definidas en el diagrama de Gantt, anteriormente hecho, que será utilizado para la planificación del progreso del proyecto. Este diagrama será una herramienta que verificará con eficiencia con el cumplimiento de los plazos establecidos y consigamos así un avance constante en todas las etapas del proyecto.

Nuestro proyecto tiene el potencial de generar impactos significativos para la sociedad, la tecnología, beneficios sociales y medioambientales. Al reconocer y detectar eventos de sonidos relevantes, podría aplicarse en mejoras en sistemas de seguridad y análisis de datos para sacar la información más relevante, un ejemplo sería en las llamadas de atención al cliente para comprobar si el cliente está satisfecho o no con la empresa, también podría ser empleado por médicos para realizar un seguimiento de sus pacientes extrayendo los sonidos relevantes y facilitando diagnósticos más precisos

En términos medioambientales nuestro proyecto podría tener un impacto significativo, por ejemplo, en la mejora de detección de sonidos, esto podría ayudar a la seguridad pública, al poder dar respuestas más rápidas y eficientes en situaciones de emergencia.

