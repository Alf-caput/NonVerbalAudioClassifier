
\section{Conclusión}
Tras realizar numerosas pruebas con distintas configuraciones, se ha logrado obtener un modelo con una reseñable precisión del 83\%. Un modelo de estas características es suficientemente robusto y fiable como para  ser implementado con éxito en diversos entornos reales, por lo que el objetivo inicial ha sido alcanzado con éxito. Cabe destacar que este clasificador solo representa la punta del iceberg del campo de la clasificación de sonidos; siendo un ámbito que aún aguarda infinitas posibilidades por explorar. Existen múltiples direcciones para futuras investigaciones y mejoras, tales como la optimización de algoritmos, la integración de técnicas de aprendizaje profundo más avanzadas, y la ampliación del conjunto de datos para incluir una mayor variedad de sonidos y contextos.

Además, a pesar de haber puesto el foco en entornos de cuidado, la aplicación de estos modelos no se limita solo a ese ámbito; pudiendo extenderse a otras áreas como la seguridad, la domótica, la asistencia personal y la industria del entretenimiento, entre muchas otras. La intersección de la clasificación de sonidos con la inteligencia artificial promete revolucionar diversos sectores, facilitando una interacción más natural y eficiente entre humanos y máquinas.


En conclusión, aunque se han alcanzado los objetivos deseados con este proyecto, continuar explorando y perfeccionando esta tecnología mejorará enormemente su precisión y aplicabilidad, abriendo nuevas fronteras de aplicación en este campo tan prometedor.
